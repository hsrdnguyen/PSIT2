\newglossaryentry{Captcha}
{
	name=CAPTCHA,
	description={
		englisch für "`Completely Automated Public Turing test to tell Computers and Humans Apart"' ("`vollautomatischer öffentlicher Turing-Test zur Unterscheidung von Computern und Menschen"'), ist ein Test, welcher Benutzer lösen müssen, um zu bestätigen, dass sie keine Maschine sind
	}
}


%\newglossaryentry{ZHAW}
%{
%	name=ZHAW,
%	description={
%		Zürcher Hochschule für Angewandte Wissenschaften ist eine technische Fachhoschle in der Schweiz%
%	}
%}

\newglossaryentry{Salt}
{
	name=Salt,
	description={ist in der Informatik eine zufällig generierte Zeichenfolge, welche zu einem Passwort  hinzugefügt wird, um bei der Verwendung eines Hash-Algorithmus aus zwei gleichen Passwörtern unterschiedliche Kennzeichnungen zu generieren}
}

\newglossaryentry{Hash-Algorithmus}
{
	name={Hash-Algorithmus},
	description={ist ein Algorithmus welcher aus einem Passwort eine nahezu eindeutige Kennzeichnung generiert}
}

\newglossaryentry{JSP}
{
	name={JavaServer-Pages},
	description={(JSP) ist eine Web-Programmiersprache zur dynamischen Erzeugung von HTML- und XML-Ausgaben eines Webservers. Sie erlaubt es Java-Code in HTML- und XML-Seiten einzubetten}
}

\newglossaryentry{utf8}
{
	name={UTF-8},
	description={ist eine Abkürzung für "`8-Bit Universal Character Set Transformation Format"' und ist eine Codierung für \gls{Unicode}-Zeichen}
}

\newglossaryentry{Unicode}
{
	name={Unicode},
	description={ist ein internationaler Standard, in dem langfristig für jedes sinntragende Schriftzeichen oder Textelement aller bekannten Schriftkulturen und Zeichensysteme ein digitaler Code festgelegt wird}
}

\newglossaryentry{TLS}
{
	name={TLS},
	description={ist eine Abkürzung für "`Transport Layer Security"' und ist ein Verschlüsselungsprotokoll zur sicheren Datenübertragung im Internet. Akuell ist die Version 1.2 von \gls{TLS}\cite{rfc5246}}
}

\newglossaryentry{LDAP}
{
	name={LDAP},
	description={ist eine Abkürzung für "`Lightweight Directory Access Protocol"' und  ist ein Netzwerkprotokoll zur Abfrage und Änderung von Informationen verteilter Verzeichnisdienste}
}

\newglossaryentry{Active Directory}
{
	name={Active Directory},
	description={ist ein von Microsoft entwickelter Verzeichnisdienst für Windows Server und unterstützt bei der Verwaltung von Benutzern und Rechten}
}

\newglossaryentry{Perfect Forward Secrecy}
{
	name={Perfect Forward Secrecy},
	description={ist die Eigenschaft von Verschlüsslungsverfahren. Werden mit \gls{TLS} gesicherte Verbindungen zwischen Server und Client aufgezeichnet und der private Schlüssel des gerät in falsche Hände, so ist es mit Perfect Forward Secrecy trotzdem nicht möglich im Nachheinein diese Verbindungen zu entschlüsseln \cite{yanzhu14whytheweb}}
}