\newglossaryentry{Captcha}
{
	name=CAPTCHA,
	description={
		englisch für "`Completely Automated Public Turing test to tell Computers and Humans Apart"'' ("`vollautomatischer öffentlicher Turing-Test zur Unterscheidung von Computern und Menschen"'), ist ein Test, welcher Benutzer lösen müssen, um zu bestätigen, dass sie keine Maschine sind
	}
}


%\newglossaryentry{ZHAW}
%{
%	name=ZHAW,
%	description={
%		Zürcher Hochschule für Angewandte Wissenschaften ist eine technische Fachhoschle in der Schweiz%
%	}
%}

\newglossaryentry{Salt}
{
	name=Salt,
	description={ist in der Informatik eine zufällig generierte Zeichenfolgen, welche zu einem Passwort  hinzugefügt wird, um bei der Verwendung eines Hash-Alorithmus aus zwei gleichen Passwörter unterschiedliche Kennzeichnungen zu generieren}
}

\newglossaryentry{Hash-Algorithmus}
{
	name={Hash-Algorithmus},
	description={ist ein Algorithmus welcher aus einem Passwort eine nahezu eindeutige Kennzeichnung generiert}
}

\newglossaryentry{JSP}
{
	name={JavaServer-Pages},
	description={abgekürzt JSP, ist eine Web-Programmiersprache zur dynamischen Erzeugung von HTML- und XML-Ausgaben eines Webservers. Sie erlaubt es Java-Code in HTML- und XML-Seiten einzubetten}
}

\newglossaryentry{utf8}
{
	name={UTF-8},
	description={ist eine Abkürzung für "`8-Bit Universal Character Set Transformation Format"' und ist eine Codierung für \gls{Unicode}-Zeichen}
}

\newglossaryentry{Unicode}
{
	name={Unicode},
	description={ist ein internationaler Standard, in dem langfristig für jedes sinntragende Schriftzeichen oder Textelement aller bekannten Schriftkulturen und Zeichensysteme ein digitaler Code festgelegt wird}
}

\newglossaryentry{TLS}
{
	name={TLS},
	description={ist eine Abkürzung für "`Transport Layer Security"' und ist ein Verschlüsselungsprotokoll zur sicheren Datenübertragung im Internet. Die neuste Version von \gls{TLS} ist die Version 1.2}
}