
\documentclass{avocado}


\usepackage[ngerman]{babel}
\usepackage[german=quotes]{csquotes}
\usepackage{float}
%\usepackage{bibgerm}
\usepackage{amsmath}
\usepackage{tabularx}
\usepackage{graphicx}
\usepackage{pdflscape}
\usepackage{xstring}
\usepackage{caption}
\usepackage[backend=biber, style=ieee]{biblatex}
\addbibresource{sources.bib}
%\usepackage[fixlanguage]{babelbib}
%\selectbiblanguage{german}
%\bibliographystyle{ieeetr}
%\addbibresource{sources}
\newcommand{\mail}[1]{\href{mailto:#1}{#1}}
\setlength{\parindent}{0cm}

\newcommand{\titel}{Avocado Share}
\newcommand{\shorttitel}{}
\newcommand{\doctype}{Pflichtenheft}
\newcommand{\untertitel}{Studentenplattform zum Know-How-Transfer}
\newcommand{\datum}{\today}
\newcommand{\team}{Gruppe 13}
\newcommand{\autorA}{Bergmann Sascha}
\newcommand{\autorB}{Kunz Lion}
\newcommand{\autorC}{Ngueyen Dang Thien}
\newcommand{\autorD}{Müller Cyril}
\newcommand{\autorE}{}
\newcommand{\ort}{Winterthur}
\newcommand{\dozent}{}
\newcommand{\betreuer}{}
\newcommand{\version}{1.6}

\hypersetup{
    bookmarks=false,    % show bookmarks bar?
    pdftitle={\titel - \doctype},    % title
    pdfauthor={\autorA, \autorB, \autorC, \autorD},  % author
    pdfsubject={Funktionale und nicht-funktionale Anforderungen}, % subject of the document
    pdfkeywords={\titel, \doctype, \team, Version \version, PSIT, IT15b}, % list of keywords
    colorlinks=true,        % false: boxed links; true: colored links
    linkcolor=blue!30!black,% color of internal links
    citecolor=black,        % color of links to bibliography
    filecolor=black,        % color of file links
    urlcolor=black,         % color of external links
    linktoc=page            % only page is linked
}%

%\project{Avocado Share}
\title{\title}
\author{\autorA \and \autorB \and \autorC \and \autorD \and \autorD}

\newcommand{\printrequirement}[2]{%
    \texttt{/#1/} #2% 
}

\newcommand{\requirementsection}[2]{%
    \section{\printrequirement{#1xxxx}{#2}}%
}


\newcommand{\requirementsubsection}[2]{%
    \subsection{\printrequirement{#1xx}{#2}}% 
}


\newcommand{\requirement}[2]{%
% #1: The requirement number e.g. "T0100"
% #2: The requirement title
    \subsubsection*{\printrequirement{#1}{#2}}%
    \expandafter\def\csname RequirementName#1\endcsname{#2}
    \expandafter\def\csname RequirementNumber#2\endcsname{#1}
    \label{subsub:requirement_#2}
}

\newcommand{\user}[1]{$U_#1$}
\newcommand{\group}[1]{$G_#1$}

\newenvironment{abnahmefall}[1][]{%
%begin{abnahmefall}
\tabularx{\textwidth}{|lX|} \hline
\ifthenelse{\equal{#1}{}}
{}{%
& \textbf{#1} \\ \hline
}%
\ignorespaces}{
%end{abnahmefall}
\endtabularx\\[1em]
}

\newcommand{\ausgangssituation}[1]{\textbf{Ausgangssituation} &\ignorespaces #1 \\ \hline}
\newcommand{\ereignis}[1]{\textbf{Ereignis} &\ignorespaces #1 \\ \hline }
\newcommand{\ergebnis}[1]{\textbf{Erwartetes Ergebnis} &\ignorespaces #1 \\ \hline}

% Refer to a requirement by it's title
\newcommand{\refreq}[1]{\nameref{subsub:requirement_#1}}
\newcommand{\tbf}{\emph{Noch zu definieren.}}

\newcommand{\abnahmekriterium}[1]{%
\StrLeft{\csname RequirementNumber#1\endcsname}{1}[\requirementsectioncharacter]%
\StrBehind{\csname RequirementNumber#1\endcsname}{\requirementsectioncharacter}[\plainrequirementnumber]%
\subsubsection*{\printrequirement{T\plainrequirementnumber}{#1}}%
}


\usepackage[toc, xindy]{glossaries}
%\newglossary[glignoredl]{ignored}{glignored}{glignoredin}{Ignored Glossary}


\makeglossaries


\begin{document}
\newglossaryentry{Captcha}
{
	name=CAPTCHA,
	description={
		englisch für "`Completely Automated Public Turing test to tell Computers and Humans Apart"'' ("`vollautomatischer öffentlicher Turing-Test zur Unterscheidung von Computern und Menschen"'), ist ein Test, welcher Benutzer lösen müssen, um zu bestätigen, dass sie keine Maschine sind
	}
}


%\newglossaryentry{ZHAW}
%{
%	name=ZHAW,
%	description={
%		Zürcher Hochschule für Angewandte Wissenschaften ist eine technische Fachhoschle in der Schweiz%
%	}
%}

\newglossaryentry{Salt}
{
	name=Salt,
	description={ist in der Informatik eine zufällig generierte Zeichenfolge, welche zu einem Passwort  hinzugefügt wird, um bei der Verwendung eines Hash-Algorithmus aus zwei gleichen Passwörtern unterschiedliche Kennzeichnungen zu generieren}
}

\newglossaryentry{Hash-Algorithmus}
{
	name={Hash-Algorithmus},
	description={ist ein Algorithmus welcher aus einem Passwort eine nahezu eindeutige Kennzeichnung generiert}
}

\newglossaryentry{JSP}
{
	name={JavaServer-Pages},
	description={(JSP) ist eine Web-Programmiersprache zur dynamischen Erzeugung von HTML- und XML-Ausgaben eines Webservers. Sie erlaubt es Java-Code in HTML- und XML-Seiten einzubetten}
}

\newglossaryentry{utf8}
{
	name={UTF-8},
	description={ist eine Abkürzung für "`8-Bit Universal Character Set Transformation Format"' und ist eine Codierung für \gls{Unicode}-Zeichen}
}

\newglossaryentry{Unicode}
{
	name={Unicode},
	description={ist ein internationaler Standard, in dem langfristig für jedes sinntragende Schriftzeichen oder Textelement aller bekannten Schriftkulturen und Zeichensysteme ein digitaler Code festgelegt wird}
}

\newglossaryentry{TLS}
{
	name={TLS},
	description={ist eine Abkürzung für "`Transport Layer Security"' und ist ein Verschlüsselungsprotokoll zur sicheren Datenübertragung im Internet. Die neuste Version von \gls{TLS} ist die Version 1.2}
}

\newglossaryentry{LDAP}
{
	name={LDAP},
	description={ist eine Abkürzung für "`Lightweight Directory Access Protocol"' und  ist ein Netzwerkprotokoll zur Abfrage und Änderung von Informationen verteilter Verzeichnisdienste.}
}

\newglossaryentry{Active Directory}
{
	name={Active Directory},
	description={ist ein von Microsoft entwickelter Verzeichnisdienst für Windows Server und unterstützt bei der Verwaltung von Benutzern und Rechten.}
}
\clearpage
\maketitle
\thispagestyle{empty}\clearpage
\section*{Dokumentenhistorie}
\begin{tabularx}{\linewidth}{|l|r|X|} \hline
Version & \multicolumn{1}{l|}{Datum} 			& Anpassungen \\ \hline
0.1 & 11.10.2015 & Dokumentstruktur \\ \hline
0.2	& 12.10.2015 & Erster Entwurf  \\ \hline
1.0 &  2.11.2015 & Erste Version, ohne Abnahmekriterien \\ \hline
1.1 & 14.11.2015 & Wiederhinzufügen der nichtfunktionalen Anforderungen \\ \hline
1.2 & 15.11.2015 & Verbesserung der Rechtschreibung und des Stils \\ \hline
1.3 & 22.11.2015 & Verbesserung der Formulieren und Strukturierung\\ \hline
1.4 & 23.11.2015 & Hinzufügen des ersten Teils der Abnahmekriterien\\ \hline
1.5 & 23.11.2015 & Kleine Änderungen und gesammte Abnahmektriterien\\ \hline
1.6 & 23.11.2015 & Logo hinzugefügt\\ \hline
\end{tabularx}

\vfill
\section*{Unterschriften}
\begin{minipage}[t][5cm][t]{0.45\linewidth}
\subsection*{Auftraggeber}
Prof. Dr. Max Lemmenmeier \\
ZHAW, Dept. Linguistik \\
%LCC Language Competence Centre \\
%Büro SF 03.09 \\
Theaterstr. 17 \\
%Postfach \\
8400 Winterthur \\
%Tel. 0041 58 934 60 73 \\
\mail{max.lemmenmeier@zhaw.ch} \\
\vfill \hrule
\end{minipage} \hfill
\begin{minipage}[t][5cm][t]{0.45\linewidth}
\subsection*{Auftragnehmer}
Leiter Projektgruppe \\
Sascha Bergmann \\
ZHAW, Bachelorstudent ICT \\
\mail{bergmsas@students.zhaw.ch} \\
\vfill \hrule
\end{minipage} \\
%\begin{minipage[t]{0.45\linewidth}
%\subsection*{Auftraggeber}
%\end{minipage}
\clearpage

\tableofcontents
\clearpage
\section{Einleitung}
\subsection{Ziel des Dokumentes}
Dieses Dokument beschreibt den Funktionsumfang und die Abgrenzungen des Avocado Share. Das Dokument dient als Grundlage für die Umsetzung des Projektes und muss sowohl vom Auftraggeber, wie auch vom Auftragnehmer unterzeichnet werden. Sobald das Dokument unterzeichnet wurde, ist es als verbindlich zu betrachten.

\section{Ziel des Avocado Share}
Die Webplattform Avocado Share soll Studentinnen und Studenten der ZHAW als Hilfe beim Austausch von Daten und Know-how dienen.
\subsection{Musskriterien}
Die Benutzer können Dateien, gruppiert nach den jeweiligen Modulen, hochladen und mit anderen teilen. Für eine vereinfachte Suche lassen sich die Dateien in Kategorien unterteilen und können bewertet werden. Die Berechtigungen lassen sich mit Gruppen, wie zum Beispiel einer Unterrichtsklasse, einfach verwalten. Wird eine solche Gruppe in ein Modul eingetragen, so haben alle Studenten und Studentinnen dieser Klasse Zugriff auf dieses Modul und die entsprechenden Dateien. Bei speziellen Berechtigungen, wenn zum Beispiel eine Person eine zusätzliche Berechtigung für ein bestimmtes Modul braucht, kann diese auch ohne eine Gruppenzugehörigkeit eingestellt werden.\\

Innerhalb der Plattform kann natürlich auch nach Kategorien und Dateien gesucht werden. Dateien, die eine hohe Bewertung haben und aktuell sind, sollen in den ersten Zeilen angezeigt werden. \\
Es soll eine Verwaltungsoberfläche geben, in der jeder einzelne Benutzer seine Benutzereinstellungen bearbeiten, anpassen und als Übersicht darstellen kann. \\
Wenn ein Benutzer eine Datei überschreibt, wird die überschriebene Datei nicht aufbewahrt. Es wird eine Versionierung geführt, um zu sehen, wer die Datei wann verändert hat, aber nicht, was verändert wurde. Somit ist es auch nicht möglich, gemachte Änderungen wieder rückgängig zu machen.
\subsection{Wunschkriterien}
Sofern die Zeit noch reicht werden noch folgende Erweiterungen, oder eine Auswahl davon, in die Applikation eingebaut.\\
Dokumente können gleich im Browser erstellt und verändert werden. Dies wird durch einen Web-Editor ermöglicht. Auch offline erstellt und dann hochgeladene Dokumente können mit diesem Editor verändert werden.\\
Sofern es die Infrastruktur erlaubt, wird der Avocado-Share über LDAP mit dem Active Directory der ZHAW verbunden, sodass jeder Student der ZHAW kein neues Login erstellen muss. Aus dem Active Directory wird auch gleich ausgelesen in welchen Klassen sich eine Person befindet und fügt diese beim ersten Login gleich den entsprechenden Gruppen hinzu.\\
\subsection{Wunschkriterien}
Sofern die Zeit noch reicht werden noch folgende Erweiterungen, oder eine Auswahl davon, in die Applikation eingebaut.\\
Dokumente können gleich im Browser erstellt und verändert werden. Dies wird durch einen Web-Editor ermöglicht. Auch offline erstellt und dann hochgeladene Dokumente können mit diesem Editor verändert werden.\\
Sofern es die Infrastruktur erlaubt, wird der Avocado-Share über LDAP mit dem Active Directory der ZHAW verbunden, sodass jeder Student der ZHAW kein neues Login erstellen muss. Aus dem Active Directory wird auch gleich ausgelesen in welchen Klassen sich eine Person befindet und fügt diese beim ersten Login gleich den entsprechenden Gruppen hinzu.\\
\subsection{Abgrenzungskriterien}
Die Applikation regelt sich nicht selbst, sprich es wird nichts automatisch gelöscht, auch wenn eine Datei veraltet ist oder nicht gebraucht wird. Für eine saubere Strukturierung der Ablage und eine klare Benennung der Dateien sind die User selbst verantwortlich. Es wird jedoch versucht durch entsprechendes Design die User zu einer guten Struktur zu bewegen. Der Avocado-Share soll auch nicht den mündlichen Know-How austausch erstezen, sondern lediglich als unterstützendes Tool dienen.
\section{Produkteinsatz}
\subsection{Anwendungsbereiche}
Der Avocado Share findet Anwendung im Alltag von Studenten verschiedenster Studienrichtungen. Da die Applikation eine plattformunabhängige Weblösung ist, ist sie standortunabhängig und mit allen internetfähigen Geräten mittels Browser nutzbar. 

\subsection{Zielgruppe}
Die Applikation richtet sich an deutschsprachige Studenten der ZHAW. Es besteht die Möglichkeit, die Plattform zu einem späteren Zeitpunkt auch für andere Schweizer Fachhochschulen und Universitäten zugänglich zu machen. Es wird besonders darauf geachtet, dass die Applikation die Bedürfnisse von Studenten der ZHAW befriedigt. Demographisch lässt sich die Zielgruppe auf Benutzer und Studierende im Alter zwischen 18 und 30 Jahren begrenzen.
\input{06_functionalrequirements}
\begin{landscape}
\section{Funktionsbaum}
\begin{figure}[H]
\centering
\includegraphics[width=0.8\linewidth]{Funktionsbaum.pdf}
\caption{Funktionsbaum}
\end{figure}
\end{landscape}
\requirementsection{R}{Nicht-funktionale Anforderungen}
Alle nicht-funktionalen Anforderungen \texttt{/R07xx/} sind aus dem Dokument "`Nicht-funktionale Anforderungen"' von Patrick Feisthammel \cite{patfeist15nifunanf} übernommen und gegebenenfalls ergänzt worden.

\requirement{R0700}{Gebrauchsfähigkeit (Usability)}
Die Webseiten müssen durch den Benutzer, welcher dem Benutzerprofil entspricht, ohne weitere Hilfe verwendet werden können. Es müssen, wenn der Benutzer diese für sein Verständnis benötigt, Hinweisfelder eingeblendet werden können.
Erweiterte Usability, wie zum Beispiel eine Erweiterungen für Blinde, wird nicht speziell berücksichtigt.

\requirement{R0700}{Fehlertoleranz}
Hinweise und Fehlermeldungen müssen für den Benutzer verständlich formuliert sein und eine Hilfestellung zur Problemlösung beinhalten. Kein auftretender Fehler darf dem Benutzer ohne Verarbeitung angezeigt werden.

\requirement{R0700}{Sprache und länderspezifische Einstellungen}
Die Webseiten sind in deutscher Sprache (Schweiz) verfasst und verwenden den Zeichensatz \gls{utf8}. Es werden die Schweiz-spezifischen Einstellungen von Datum, Zeit, Zahlen und Währung verwendet.

\requirement{R0700}{Zielplattform (Server)}
Die Web-Applikation muss als \gls{JSP} auf dem zur Verfügung gestellten virtuellen Server unter Verwendung einer SQL-Datenbank mit Apache Tomcat betrieben werden.

\requirement{R0700}{Zielplattform (Client)}
Die Webseiten werden in der aktuellsten freigegebenen Version des Mozilla Firefox und Google Chrome korrekt dargestellt. Die Webseite muss für Bildschirmgrössen mit einer Breite ab 480 Pixel problemlos darstellbar sein. Für Bildschirme mit einer Breite unter 600 Pixel muss eine Mobilansicht bereitstehen. Die Webseite muss skalierbar sein und sich an die Bildschrirmbreite anpassen.

\requirement{R0700}{Werkzeuge zur Entwicklung}
Als Projektmanagement-Tool und zur Verwaltung des Sourcecodes und der Dokumente muss der zur Verfügung gestellte Github-Server verwendet werden.

\requirement{R0700}{Robustheit}
Auch nach einem Neustart des virtuellen Servers muss die Webseite voll funktionsfähig sein.

\requirement{R0700}{Testbarkeit}
Für die Durchführung der Tests und der Abnahme müssen sinnvolle Testdaten in genügendem Umfang zur Verfügung gestellt werden.

\requirementsubsection{R08}{Weitere Anforderungen}
\requirement{R0800}{Sicherheit}
Die Benutzerdaten und Dateien müssen mit einem Mindestmass an Sicherheit geschützt sein. Das heisst, dass Unbefugte nur unter einem grossen Aufwand an sicherheitsrelevante Daten kommen. Benutzerpasswörter werden nicht im Klartext gespeichert, sondern sie werden mit einem Salt kombiniert und mit Hilfe eines starken, kryptographischen \gls{Hash-Algorithmus}, wie SHA-256 oder SHA-512, in einen Hash umgewandelt. Hochgeladene Dateien werden unverschlüsselt abgespeichert.

\requirement{R0810}{Verbindungssicherheit}
Um das Abhören von Passwörtern möglichst zu vermeiden, muss die HTTP-Verbindung zwischen Client und Server mit \gls{TLS} 1.2 gesichert werden. Der verwendete Algorithmus muss \gls{Perfect Forward Secrecy} unterstützen. Das Serverzertifikat muss dabei nicht von Zertifizierungsstellen verifiziert sein.

\requirement{R0820}{Antwortzeit}
Eine browserseitige Anfrage auf eine Webseite muss, falls in folgendem Text nicht anders definiert, vom Server innerhalb von maximal einer Sekunden bearbeitet und beantwortet werden. In die Antwortzeit wird weder die Übertragungszeit noch die Zeit, die der Browser benötigt, um die Webseite anzuzeigen, eingerechnet. Anfragen, eine Datei herunterzuladen (siehe \refreq{Datei anzeigen}), dürfen maximal 5 Sekunden dauern, bis der Download startet. Eine Suche muss innerhalb von maximal 5 Sekunden beantwortet werden. Aktionen, welche eine längere Rechenzeit auf dem Server beanspruchen, müssen asynchron zur Webseiten-Anfrage bearbeitet werden.


\iffalse
\begin{table}[H]
\centering
\begin{tabular}{|l|l|} \hline
\textbf{Seitenaufruf} & \textbf{Maximale Antwortzeit} & \textbf{Funktionale Anforderung}\\ \hline
Allgemein 					& 1 Sekunde  & \\ \hline
Start eines Datei-Downloads & 5 Sekunden & \refreq{Datei anzeigen}\\ \hline
Suche						& 5 Sekunden & \refreq{Suche}\\ \hline
\end{tabular}
\caption{Maximale Antwortzeit von Seitenaufrufen.}
\end{table}
\fi

\requirement{R0830}{Wiederherstellbarkeit}
Von den Systemdateien und den Dateien der Benutzer müssen von Zeit zu Zeit Sicherungskopien erstellt werden, sodass bei einem Dateiverlust im Hauptsystem ein Grossteil der verlorenen Daten wieder hergestellt werden kann. 

\requirement{R0840}{Erweiterbarkeit}
Die Systemkomponenten müssen so implementiert sein, dass diese zu einem späteren Zeitpunkt ohne grossen Mehraufwand erweitert und auf neue Bedürfnisse abgestimmt werden können. Texte müssen zentral gespeichert werden, dass sie einfach in andere Sprachen übersetzt werden könne.

\requirement{R0850}{Dokumentation in Quellcode}
Im Quellcode müssen alle Methoden, Klassen und Datenfelder dokumentiert werden.
%%%%%%%%%%%%%%%%%%%%%%%%%%%%%%%%%%%%%%%%%%%%%%%%%%%%%%%%%%%%%%%%%%%%%%%%%%%%%%%
\requirementsection{T}{Abnahmekriterien}
%%%%%%%%%%%%%%%%%%%%%%%%%%%%%%%%%%%%%%%%%%%%%%%%%%%%%%%%%%%%%%%%%%%%%%%%%%%%%%%

%%%%%%%%%%%%%%%%%%%%%%%%%%%%%%%%%%%%%%%%%%%%%%%%%%%%%%%%%%%%%%%%%%%%%%%%%%%%%%%
\requirementsubsection{T01}{Sicherheit}
%%%%%%%%%%%%%%%%%%%%%%%%%%%%%%%%%%%%%%%%%%%%%%%%%%%%%%%%%%%%%%%%%%%%%%%%%%%%%%%

%%%%%%%%%%%%%%%%%%%%%%%%%%%%%%%%%%%%%%%%%%%%%%%%%%%%%%%%%%%%%%%%%%%%%%%%%%%%%%%
\abnahmekriterium{Zugriffskontrolle}
%%%%%%%%%%%%%%%%%%%%%%%%%%%%%%%%%%%%%%%%%%%%%%%%%%%%%%%%%%%%%%%%%%%%%%%%%%%%%%%

\begin{abnahmefall}[Zugriff eines nicht angemeldeten Benutzers]
\ausgangssituation{
Ein registrierter Benutzer ist nicht angemeldet.
}
\ereignis{
Der Benutzer versucht irgend eine Seite der Applikation zu öffnen, abgesehen von der Seite zur Registrierung (/F0200/ Registrierung) und der Login Seite (\refreq{An- und Abmelden}).
}
\ergebnis{
Der Benutzer wird auf die Login Seite umgeleitet und es wird eine Fehlermeldung angezeigt, dass er angemeldet sein muss um diese Seite zu öffnen.
}
\end{abnahmefall}
%%%%%%%%%%%%%%%%%%%%%%%%%%%%%%%%%%%%%%%%%%%%%%%%%%%%%%%%%%%%%%%%%%%%%%%%%%%%%%%
\begin{abnahmefall}[Zugriff eines angemeldeten Benutzers mit Berechtigung]
	\ausgangssituation{
		Ein angemeldeter Benutzer und besitzt die erforderlichen Berechtigungen eine Seite aufzurufen. Beispielsweise besitzt er das Vewaltungsrecht einer Gruppe, dann hat er Zugriffsberechtigung auf die Seite zum Editieren der Eigenschaften dieser Gruppe.
	}
	\ereignis{
		Der Benutzer versucht irgendeine Seite der Applikation zu öffnen, auf welche er die Zugriffsberechtigung hat.
	}
	\ergebnis{
		Die vom Benutzer aufgerufene Seite wird angzeigt.
	}
\end{abnahmefall}
%%%%%%%%%%%%%%%%%%%%%%%%%%%%%%%%%%%%%%%%%%%%%%%%%%%%%%%%%%%%%%%%%%%%%%%%%%%%%%%
\begin{abnahmefall}[Zugriff eines angemeldeten Benutzers ohne Berechtigung]
	\ausgangssituation{
		Ein angemeldeter Benutzer hat keine Zugriffsberechtigung auf die Seite hat, die er öffnen will. Hat er zum Beispiel nur Leserecht an einer Gruppe, so hat er keine Zugriffsberechtigung auf die Seite zum Editieren der Eigenschaften einer Gruppe.
	}
	\ereignis{
		Der Benutzer versucht irgendeine Seite der Applikation zu öffnen, auf welche er keine Zugriffsberechtigung hat.
	}
	\ergebnis{
		Die vom Benutzer aufgerufene Seite wird nicht angezeigt und es wird eine Fehlermeldung angezeigt, dass der Benutzer die benötigten Rechte zum Aufrufen der Seite nicht besitzt.
	}
\end{abnahmefall}
%%%%%%%%%%%%%%%%%%%%%%%%%%%%%%%%%%%%%%%%%%%%%%%%%%%%%%%%%%%%%%%%%%%%%%%%%%%%%%%
\begin{abnahmefall}[Verwaltungsrecht erteilen/entziehen]
	\ausgangssituation{
		Ein angemeldeter Benutzer \user{1} besitzt Verwaltungsrecht an einem Objekt.
		Es existiert ein anderer Benutzer \user{2}, der auch Verwaltungsrecht an diesem Objekt besitzt.
		Keiner der beiden Benutzer (\user{1} oder \user{2}) ist der Ersteller des Objektes.
	}
	\ereignis{
		Der Benutzer \user{1} erteilt einem anderen Benutzer \user{3} Verwaltungsrecht am Objekt und entzieht dem Benutzer \user{2} das Verwaltungsrecht am Objekt.
	}
	\ergebnis{
		Der Benutzer \user{3} hat nun Verwaltungsrecht am erwähnten Objekt.
		Der Versuch zum Entziehen der Verwaltungsrechte des Benutzers \user{2} scheiter und es wird eine Fehlermeldung ausgegeben, da der Benutzer \user{1} nicht Ersteller des Objektes ist.
	}
\end{abnahmefall}
%%%%%%%%%%%%%%%%%%%%%%%%%%%%%%%%%%%%%%%%%%%%%%%%%%%%%%%%%%%%%%%%%%%%%%%%%%%%%%%
\begin{abnahmefall}[Verwaltungsrecht erteilen/entziehen]
	\ausgangssituation{
		Ein angemeldeter Benutzer \user{1} besitzt Verwaltungsrecht an einem Objekt.
		Es existiert ein anderer Benutzer \user{2}, der auch Verwaltungsrecht an
		diesem Objekt besitzt. Keiner der beiden Benutzer (\user{1} oder \user{2}) ist
		der Ersteller des Objektes.
	}
	\ereignis{
		Der Benutzer \user{1} erteilt einem anderen Benutzer \user{3} Verwaltungsrecht
		am Objekt und entzieht dem Benutzer \user{2} das Verwaltungsrecht am Objekt.
	}
	\ergebnis{
		Der Benutzer \user{3} hat nun Verwaltungsrecht am erwähnten Objekt. Der
		Versuch zum Entziehen der Verwaltungsrechte des Benutzers \user{2}
		scheiter und es wird eine Fehlermeldung ausgegeben, da der
		Benutzer \user{1} nicht Ersteller des Objektes ist.
	}
\end{abnahmefall}
%%%%%%%%%%%%%%%%%%%%%%%%%%%%%%%%%%%%%%%%%%%%%%%%%%%%%%%%%%%%%%%%%%%%%%%%%%%%%%%
\abnahmekriterium{An- und Abmelden}
%%%%%%%%%%%%%%%%%%%%%%%%%%%%%%%%%%%%%%%%%%%%%%%%%%%%%%%%%%%%%%%%%%%%%%%%%%%%%%%
\begin{abnahmefall}[Anmelden]
	\ausgangssituation{
		Ein registrierter Benutzer ist nicht angemeldet.
	}
	\ereignis{
		Der Benutzer öffnet die Applikation, es wird ein Anmeldeformular
		angezeigt, in welches der Benutzer seine E-Mail-Adresse und sein
		Passwort eingibt.
	}
	\ergebnis{
		Falls seine Eingaben korrekt sind, wird der Benutzer angemeldet
		und auf die Hauptseite umgeleitet. Falls die Eingaben fehlerhaft
		sind, wird eine Fehlermeldung angezeigt und der Benutzer kann es
		erneut versuchen. Zudem wird ein Link angezeigt um sein Passwort
		zurückzusetzten.
	}
\end{abnahmefall}
%%%%%%%%%%%%%%%%%%%%%%%%%%%%%%%%%%%%%%%%%%%%%%%%%%%%%%%%%%%%%%%%%%%%%%%%%%%%%%%
\begin{abnahmefall}[Anmelden]
	\ausgangssituation{
		Ein Benutzer ist angemeldet.
	}
	\ereignis{
		Der angemeldete Benutzer meldet sich über eine Schaltfläche ab.
	}
	\ergebnis{
		Der Benutzer ist abgemeldet und wird auf die Startseite der Applikation umgeleitet.
	}
\end{abnahmefall}
%%%%%%%%%%%%%%%%%%%%%%%%%%%%%%%%%%%%%%%%%%%%%%%%%%%%%%%%%%%%%%%%%%%%%%%%%%%%%%%
\abnahmekriterium{Passwort zurücksetzen}
%%%%%%%%%%%%%%%%%%%%%%%%%%%%%%%%%%%%%%%%%%%%%%%%%%%%%%%%%%%%%%%%%%%%%%%%%%%%%%%
\begin{abnahmefall}[]
	\ausgangssituation{
		Ein bestehender Benutzer, welcher aus der Applikation abgemeldet ist und
		sich anmelden will, weiss sein Passwort nicht mehr.
	}
	\ereignis{
		Der abgemeldete Benutzer gibt in das Passwort zurücksetzten Formular,
		die für seinen Benutzer zutreffende E-Mail-Adresse ein und füllt danach
		das \gls{Captcha} aus und lässt sich ein neues Passwort generieren.
	}
	\ergebnis{
		Der Benutzer ist abgemeldet und wird auf die Startseite der Applikation umgeleitet.
	}
\end{abnahmefall}
%%%%%%%%%%%%%%%%%%%%%%%%%%%%%%%%%%%%%%%%%%%%%%%%%%%%%%%%%%%%%%%%%%%%%%%%%%%%%%%
\requirementsubsection{T02}{Benutzerverwaltung}
%%%%%%%%%%%%%%%%%%%%%%%%%%%%%%%%%%%%%%%%%%%%%%%%%%%%%%%%%%%%%%%%%%%%%%%%%%%%%%%
%%%%%%%%%%%%%%%%%%%%%%%%%%%%%%%%%%%%%%%%%%%%%%%%%%%%%%%%%%%%%%%%%%%%%%%%%%%%%%%
\abnahmekriterium{Registrierung}
%%%%%%%%%%%%%%%%%%%%%%%%%%%%%%%%%%%%%%%%%%%%%%%%%%%%%%%%%%%%%%%%%%%%%%%%%%%%%%%
\begin{abnahmefall}[Registrierung]
	\ausgangssituation{
		Ein Benutzer ist noch nicht im System erfasst und will sich neu registrieren.
	}
	\ereignis{
		Der Benutzer gibt im Registrierungs-Formular seinen Vor- und Nachnamen, seine E-Mail-Adresse und sein gewünschtes Passwort ein und schliesst die Registrierung über die entsprechende Schaltfläche ab.
	}
	\ergebnis{
		Falls alle Angaben gemacht wurden und die E-Mail-Adresse ein gültiges Format hat, wird dem Benutzer eine E-Mail zur Bestätigung gesendet, mit welcher er sein Konto aktivieren kann. Falls die Angaben nicht korrekt waren, wird das Formular nochmals angezeigt und eine Fehlermeldung ausgegeben. In der Bestätigungs-E-Mail ist ein Link zum Aktivieren des Benutzerkontos enthalten und eine kurze Beschreibung über den Nutzen dieses Links.
	}
\end{abnahmefall}
%%%%%%%%%%%%%%%%%%%%%%%%%%%%%%%%%%%%%%%%%%%%%%%%%%%%%%%%%%%%%%%%%%%%%%%%%%%%%%%
\begin{abnahmefall}[Registrierung bestätigen]
	\ausgangssituation{
		Ein Benutzer hat die Registrierung abgeschlossen hat und bekam ein E-Mail, in welchem ein Link zum Aktivieren des Benutzerkontos enthalten ist.
	}
	\ereignis{
		Der Benutzer klickt im E-Mail auf den Link zur Aktivierung.
	}
	\ergebnis{
		Beim Öffnen des Links, wird man auf die Applikations-Seite weitergeleitet und der entsprechende Benutzer freigeschaltet. Dem Benutzer wird nun ein Loginformular und eine Bestätigung der Aktivierung angezeigt.
	}
\end{abnahmefall}
%%%%%%%%%%%%%%%%%%%%%%%%%%%%%%%%%%%%%%%%%%%%%%%%%%%%%%%%%%%%%%%%%%%%%%%%%%%%%%%
\abnahmekriterium{Benutzerdaten ändern}
%%%%%%%%%%%%%%%%%%%%%%%%%%%%%%%%%%%%%%%%%%%%%%%%%%%%%%%%%%%%%%%%%%%%%%%%%%%%%%%
\begin{abnahmefall}[Benutzerdaten ändern]
	\ausgangssituation{
		Ein bestehender Benutzer ist in der Applikation angemeldet.
	}
	\ereignis{
		Der Benutzer editiert über ein Formular seine Benutzerdaten, genauer gesagt seinen Vor- und Nachnamen, sein Profilbild, seine Berechtigungen und seine E-Mail-Adresse und speichert diese ab.
	}
	\ergebnis{
		 Abgesehen von der E-Mail-Adresse wurden alle Änderungen übernommen. Für die Änderung an der E-Mail-Adresse wurde an die neu eingegebene Adresse ein Bestätigungs-E-Mail gesandt, in welchem ein Link zum Bestätigen der E-Mail-Adresse enthalten ist.
	}
\end{abnahmefall}
%%%%%%%%%%%%%%%%%%%%%%%%%%%%%%%%%%%%%%%%%%%%%%%%%%%%%%%%%%%%%%%%%%%%%%%%%%%%%%%
\begin{abnahmefall}[Änderung an E-Mail bestätigen]
	\ausgangssituation{
		Ein Benutzer hat im Benutzerdaten-Änderungsformular seine E-Mail Adresse geändert und es wurde ihm an die neu eingetragene E-Mail-Adresse eine Bestätigungs-E-Mail gesandt.
	}
	\ereignis{
		Der Benutzer klickt im E-Mail auf den Link zum Validieren der geänderten E-Mail-Adresse.
	}
	\ergebnis{
		 Die E-Mail-Adresse des Benutzers wird geändert und die Startseite der Applikation wird geöffnet.
	}
\end{abnahmefall}
%%%%%%%%%%%%%%%%%%%%%%%%%%%%%%%%%%%%%%%%%%%%%%%%%%%%%%%%%%%%%%%%%%%%%%%%%%%%%%%
\begin{abnahmefall}[Benutzerkonto löschen]
	\ausgangssituation{
		Ein bestehender Benutzer ist in der Applikation angemeldet.
	}
	\ereignis{
		Der Benutzer löscht sein Benutzerkonto über eine entsprechende Schaltfläche.
	}
	\ergebnis{
		 Der Benutzer wird abgemeldet und auf die Startseite der Applikation weitergeleitet. Sein Benutzerkonto ist nun gelöscht und wenn der Benutzer versucht sich wieder anzumelden, ist dies nicht möglich.
	}
\end{abnahmefall}
%%%%%%%%%%%%%%%%%%%%%%%%%%%%%%%%%%%%%%%%%%%%%%%%%%%%%%%%%%%%%%%%%%%%%%%%%%%%%%%
\requirementsubsection{T03}{Gruppenverwaltung}
%%%%%%%%%%%%%%%%%%%%%%%%%%%%%%%%%%%%%%%%%%%%%%%%%%%%%%%%%%%%%%%%%%%%%%%%%%%%%%%
\abnahmekriterium{Gruppe erstellen}
%%%%%%%%%%%%%%%%%%%%%%%%%%%%%%%%%%%%%%%%%%%%%%%%%%%%%%%%%%%%%%%%%%%%%%%%%%%%%%%
\begin{abnahmefall}[Gruppe erstellen]
	\ausgangssituation{
		Ein bestehender Benutzer ist in der Applikation angemeldet.
	}
	\ereignis{
		Der Benutzer erstellt über ein Formular eine neue Gruppe und  konfiguriert diese Gruppe (siehe Gruppe bearbeiten) sogleich. Er bestimmt lediglich den Namen der Gruppe und schliesst das Erstellen der Gruppe ab.
	}
	\ergebnis{
		Falls keine andere Gruppe mit dem gleichen Namen bereits existiert, wird die Gruppe mit den eingegebenen Konfigurationen erstellt. Zusätzlich wird der Benutzer, welcher die Gruppe erstellt hat, gleich als Ersteller der Gruppe eingetragen, das Erstelldatum der Gruppe wird gesetzt und der Ersteller bekommt sogleich Verwaltungs-Rechte an der Gruppe. Ist der Gruppenname bereits vergeben wird dem Benutzer eine Fehlermeldung, dass bereits eine Gruppe mit dem gleichen Namen existiert, angezeigt und das Erstellformular für die Gruppe wird erneut angezeigt.
	}
\end{abnahmefall}
%%%%%%%%%%%%%%%%%%%%%%%%%%%%%%%%%%%%%%%%%%%%%%%%%%%%%%%%%%%%%%%%%%%%%%%%%%%%%%%
\abnahmekriterium{Gruppe bearbeiten}
%%%%%%%%%%%%%%%%%%%%%%%%%%%%%%%%%%%%%%%%%%%%%%%%%%%%%%%%%%%%%%%%%%%%%%%%%%%%%%%
\begin{abnahmefall}[Gruppe bearbeiten]
	\ausgangssituation{
		Es gibt eine Gruppe mit mehreren Mitgliedern. Ein angemeldeter Benutzer besitzt Verwaltungsrecht an dieser Gruppe.
	}
	\ereignis{
		Der Benutzer ändert über das Bearbeitungs-Formular der Gruppe den Namen, die Beschreibung und die Mitglieder und speichert diese Änderungen ab.
	}
	\ergebnis{
		Falls der geänderte Name nicht gleich wie der einer anderen Gruppe lautet, werden alle Änderungen an der Gruppe übernommen und die Seite mit den Eigenschaften der Gruppe angezeigt. Ansonsten wird eine Fehlermeldung ausgegeben und das Bearbeitungs-Formular wird wieder angezeigt.
	}
\end{abnahmefall}
%%%%%%%%%%%%%%%%%%%%%%%%%%%%%%%%%%%%%%%%%%%%%%%%%%%%%%%%%%%%%%%%%%%%%%%%%%%%%%%
\begin{abnahmefall}[Gruppe löschen]
	\ausgangssituation{
		Es gibt eine Gruppe mit mehreren Mitgliedern. Ein angemeldeter Benutzer besitzt Verwaltungsrecht an dieser Gruppe.
	}
	\ereignis{
		Der Benutzer ändert über das Bearbeitungs-Formular der Gruppe den Namen, die Beschreibung und die Mitglieder und speichert diese Änderungen ab.
	}
	\ergebnis{
		Falls der geänderte Name nicht gleich wie der einer anderen Gruppe lautet, werden alle Änderungen an der Gruppe übernommen und die Seite mit den Eigenschaften der Gruppe angezeigt. Ansonsten wird eine Fehlermeldung ausgegeben und das Bearbeitungs-Formular wird wieder angezeigt.
	}
\end{abnahmefall}
%%%%%%%%%%%%%%%%%%%%%%%%%%%%%%%%%%%%%%%%%%%%%%%%%%%%%%%%%%%%%%%%%%%%%%%%%%%%%%%
\abnahmekriterium{Gruppenrechte bearbeiten}
%%%%%%%%%%%%%%%%%%%%%%%%%%%%%%%%%%%%%%%%%%%%%%%%%%%%%%%%%%%%%%%%%%%%%%%%%%%%%%%
\begin{abnahmefall}[Gruppen-Berechtigungen erteilen]
	\ausgangssituation{
		Es existiert die Gruppen \group{1}, \group{2} und \group{3}, ein angemeldeter Benutzer \user{1} und die Benutzer \user{2} und \user{3}. Des weiteren gibt es Objekte an denen die Gruppe \group{1} Leserecht, Schreibrecht oder Verwaltungsrecht besitzt. Die Tabelle~\ref{tab:ausgangssituation_grippenrechte_erteilen} zeigt die Rechte der Gruppen und Benutzer an der Gruppe \group{1}.
		\\[1em]
		&
		\begin{tabular}{|l|l|} \hline
			\textbf{Inhaber des Rechts} & \textbf{Rechte an der Gruppe \group{1}} \\ \hline
			Benutzer \user{1} 			& Verwaltungsrecht \\ \hline
			Gruppe \group{2}, Benutzer \user{2} 	& Keine Rechte \\ \hline
			Gruppe \group{3}, Benutzer \user{3}  & Leserecht \\ \hline 
		\end{tabular}
		\captionof{table}{Rechte der Benutzer und Gruppen in der Gruppe \group{1}}
		\label{tab:ausgangssituation_grippenrechte_erteilen}
	}%
	\ereignis{
		Der Benutzer \user{1} erteilt dem Benutzer \user{2} und der Gruppe \group{2} Leserechte an der Gruppe \group{1}. Dem Benutzer \user{3} und der Gruppe \group{3} erteilt der Benutzer \user{1} Verwaltungsrecht an der Gruppe \group{1}.
	}%
	\ergebnis{
		Der Benutzer \user{2} und die Gruppe \group{2} übernehmen alle Berechtigungen welche die Gruppe (\group{1}) in anderen Modulen, Gruppen und Dateien besitzt. Die Gruppe \group{3} und der Benutzer \user{3} haben nun die Verwaltungsrecht an der Gruppe \group{1} und können nun deren Eigenschaften ändern (siehe Gruppe bearbeiten). 
	}
\end{abnahmefall}
%%%%%%%%%%%%%%%%%%%%%%%%%%%%%%%%%%%%%%%%%%%%%%%%%%%%%%%%%%%%%%%%%%%%%%%%%%%%%%%
\begin{abnahmefall}[Gruppen-Berechtigungen entziehen]
	\ausgangssituation{
		Es existieren die Gruppen \group{1}, \group{2} und \group{3}, ein angemeldeter Benutzer \user{1}, welcher nicht der Ersteller der Gruppe \group{1} ist, und die Benutzer \user{2} und \user{3}. Die nachfolgende Tabelle~\ref{tab:ausgangssituation_grippenrechte_entziehen} zeigt die Berechtigungen der Benutzer und Gruppen an der Gruppe \group{1}.\\[1em]
		&
		%\begin{table}[H]
		\begin{tabular}{|l|l|} \hline
		\textbf{Inhaber des Rechts} & \textbf{Rechte an der Gruppe \group{1}}\\ \hline
		Benutzer \user{1} 			& Verwaltungsrecht \\ \hline
		Gruppe \group{2}, Benutzer \user{2} 	& Leserecht \\ \hline
		Gruppe \group{3}, Benutzer \user{3}  & Verwaltungsrecht \\ \hline 
		\end{tabular}
		\captionof{table}{Rechte der Benutzer und Gruppen in der Gruppe \group{1}}
		\label{tab:ausgangssituation_grippenrechte_entziehen}
		%\label{tab:ausgangssituation_grippenrechte_entziehen}
		%\end{table}
	}%
	\ereignis{
		Der Benutzer \user{1} entzieht den Gruppen \group{2} und \group{2} und den Benutzern \user{2} und \user{3} alle Rechte an der Gruppe \group{1}.
	}%
	\ergebnis{
		Der Gruppe \group{2} und dem Benutzer \user{2} werden alle Berechtigungen an der Gruppe \group{1} entzogen und somit auch alle Berechtigungen an anderen Modulen, Gruppe und Dateien, welche mit der Mitgliedschaft an der Gruppe \group{1} einhergegangen sind. Der Gruppen \group{3} und Benutzer \user{3} wird das Verwaltungsrecht an der Gruppe \group{1} nicht entzogen und dem Benutzer \user{1} wird eine Fehlermeldung ausgegeben, da der Benutzer \user{1} diese Berechtigung nicht entziehen kann.
	}%
\end{abnahmefall}
%%%%%%%%%%%%%%%%%%%%%%%%%%%%%%%%%%%%%%%%%%%%%%%%%%%%%%%%%%%%%%%%%%%%%%%%%%%%%%%

\clearpage

\glsaddall
\printglossaries

%\bibliographystyle{ieeetr}
\nocite{*}
%\bibliography{sources}
\printbibliography[heading=bibintoc]

\end{document}