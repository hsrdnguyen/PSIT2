\section{Ziel des Avocado Share}
Die Webplattform Avocado Share soll Studentinnen und Studenten der ZHAW als Hilfe beim Austausch von Daten und Know-how dienen.
Die Benutzer können Dateien gruppiert nach den jeweiligen Modulen hochgeladen und mit anderen geteilt werden. Für eine vereinfachte Suche lassen sich die Dateien in Kategorien unterteilen und können bewertet werden. Die Berechtigungen lassen sich mit Gruppen, wie zum Beispiel einer Unterrichtsklasse, einfach verwalten. Wird eine solche Gruppe in ein Modul eingetragen, so haben alle Studenten und Studentinnen dieser Klasse Zugriff auf dieses Modul und die entsprechenden Dateien. Bei speziellen Berechtigungen, wenn zum Beispiel eine Person eine zusätzliche Berechtigung für ein bestimmtes Modul braucht, kann dies auch ohne Gruppen eingestellt werden.\\

Innerhalb der Plattform kann natürlich auch nach Kategorien und Dateien gesucht werden. Dateien, die eine hohe Bewertung haben und aktuell sind, sollen in den ersten Zeilen angezeigt werden. \\
Es soll eine Verwaltungsoberfläche geben, in der jeder einzelne Benutzer seine Benutzereinstellungen bearbeiten, anpassen und als Übersicht darstellen kann. \\
Wenn ein Benutzer eine Datei überschreibt, wird die überschriebene Datei nicht aufbewahrt. Es wird eine Versionierung geführt um zu sehen, wer die Datei wann verändert hat, aber nicht, was verändert wurde. Somit ist es auch nicht möglich, gemachte Änderungen wieder rückgängig zu machen.