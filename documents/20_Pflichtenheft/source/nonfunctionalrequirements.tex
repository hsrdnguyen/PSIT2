\requirementsection{R}{Nicht-funktionale Anforderungen}
Alle nicht-funktionalen Anforderungen /R05xx/ sind aus dem Dokument “Nicht-funktionale Anforderungen” von Patrick Feisthammel \cite{patfeist15nifunanf} übernommen und gegebenenfalls ergänzt worden.

\setcounter{RequirementSubsectionCounter}{5}
\requirement{00}{Gebrauchsfähigkeit (Usability)}
Die Webseiten müssen durch den Benutzer, welcher dem Benutzerprofil entspricht, ohne weitere Hilfe verwendet werden können. Es müssen, wenn der Benutzer diese für sein Verständnis benötigt, Hinweisfelder eingeblendet werden können.
Erweiterte Usability, wie zum Beispiel Erweiterungen für Blinde, wird nicht speziell berücksichtigt.

\requirement{10}{Fehlertoleranz}
Hinweise und Fehlermeldungen müssen für den Benutzer verständlich formuliert sein und eine Hilfestellung zur Problemlösung beinhalten. Kein auftretender Fehler darf ohne Verarbeitung dem Benutzer angezeigt werden.

\requirement{20}{Sprache und länderspezifische Einstellungen}
Die Webseiten sind in deutscher Sprache (Schweiz) verfasst, verwenden den Zeichensatz \gls{utf8}. Es werden die Schweiz-spezifischen Einstellungen von Datum, Zeit, Zahlen und Währung verwendet.

\requirement{30}{Zielplattform (Server)}
Die Web-Applikation muss als \gls{JSP} auf dem zur Verfügung gestellten virtuellen Server unter Verwendung einer SQL-Datenbank mit Apache Tomcat betrieben werden.

\requirement{40}{Zielplattform (Client)}
Die Webseiten werden in der aktuellsten freigegebenen Version des Mozilla Firefox und Google Chrome korrekt dargestellt. Die Webseite muss für Bildschirmgrössen mit einer Breite ab 480 Pixel problemlos darstellbar sein. Für Bildschirme mit einer Breite unter 600 Pixel muss eine Mobilansicht bereitstehen. Die Webseite muss skalierbar sein und sich an die Bildschrirmbreite anpassen.

\requirement{50}{Werkzeuge zur Entwicklung}
Als Projektmanagement-Tool und zur Verwaltung des Sourcecodes und der Dokumente muss der zur Verfügung gestellte Github- Server verwendet werden.

\requirement{60}{Robustheit}
Auch nach einem Neustart des virtuellen Servers muss die Webseite voll funktionsfähig sein.

\requirement{70}{Testbarkeit}
Für die Durchführung der Tests und der Abnahme müssen sinnvolle Testdaten in genügendem Umfang zur Verfügung gestellt werden.

\requirementsubsection{Weitere Anforderungen}
\requirement{0}{Sicherheit}
Die Benutzerdaten und Dateien müssen mit einem Mindestmass an Sicherheit geschützt sein. Das heisst, dass Unbefugte nur unter einem grossen Aufwand an sicherheitsrelevante Daten kommen. Benutzerpasswörter werden nicht im Klartext gespeichert, sondern sie werden mit einem Salt kombiniert und mit Hilfe eines starken, kryptographischen \gls{Hash-Algorithmus}, wie SHA-256 oder SHA-512, in einen Hash umgewandelt. Hochgeladene Dateien werden unverschlüsselt abgespeichert.

\requirement{10}{Verbindungssicherheit}
Um das Abhören von Passwörtern möglichst zu vermeiden, muss die HTTP-Verbindung zwischen Client und Server mit \gls{TLS} 1.2 gesichert werden. Das Serverzertifikat muss dabei nicht von Zertifizierungsstellen verifiziert sein.

\requirement{20}{Antwortzeit}
Eine browserseitige Anfrage auf eine Webseite muss, falls in folgendem Text nicht anders definiert, vom Server innerhalb von maximal einer Sekunden bearbeitet und beantwortet werden. In die Antwortzeit wird weder die Übertragungszeit noch die Zeit, die der Browser benötigt, um die Webseite anzuzeigen, eingerechnet. Anfragen, eine Datei herunterzuladen (siehe Datei herunterladen), dürfen maximal 5 Sekunden dauern, bis der Download startet. Eine Suche muss innerhalb von maximal 5 Sekunden beantwortet werden. Aktionen, welche eine längere Rechenzeit auf dem Server beanspruchen, müssen asynchron zur Webseiten-Anfrage bearbeitet werden.

\requirement{30}{Wiederherstellbarkeit}
Von den Systemdateien und den Dateien der Benutzer müssen von Zeit zu Zeit Sicherungskopien erstellt werden, sodass bei einem Dateiverlust im Hauptsystem ein Grossteil der verlorenen Daten wieder hergestellt werden kann. 

\requirement{40}{Erweiterbarkeit}
Die Systemkomponenten müssen so implementiert sein, dass diese zu einem späteren Zeitpunkt ohne grossen Mehraufwand erweitert und auf neue Bedürfnisse abgestimmt werden können. Texte müssen zentral gespeichert werden, dass sie einfach in andere Sprachen übersetzt werden könne.

\requirement{50}{Dokumentation in Quellcode}
Im Quellcode müssen alle Methoden, Klassen und Datenfelder dokumentiert werden.