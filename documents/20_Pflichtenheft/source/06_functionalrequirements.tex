\requirementsection{F}{Funktionale Anforderungen}

\requirementsubsection{F01}{Sicherheit}

\requirement{F0100}{Zugriffskontrolle}
Beim Aufrufen der verschiedenen Seiten prüft das System immer,
ob der Benutzer angemeldet ist. Falls er nicht angemeldet ist,
wird er auf die Log-in Seite verwiesen. Davon ausgenommen sind
nur die Seiten zur Registrierung (\refreq{Registrierung}) und diejenige zur Anmeldung (\refreq{An- und Abmelden}). \\

Auf alle Module, Gruppen und Dateien hat jeder Benutzer individuelle Zugriffsrechte. Folgende Tabelle~\ref{tab:rights} zeigt, welche grundsätzlichen Berechtigungen es gibt.

\begin{table}[H]
\begin{tabularx}{\textwidth}{|l|l|X|} \hline
\textbf{Stufe} & \textbf{Berechtigung}     & \textbf{Beschreibung} \\ \hline
0     & Keine Rechte     & Der Benutzer hat keine Erlaubnis, den Inhalt zu betrachten.\\ \hline
1     & Leserecht        & Der Benutzer hat die Erlaubnis, den Inhalt zu betrachten.\\ \hline
2     & Schreibrecht     & Der Benutzer hat die Erlaubnis, den Inhalt zu verändern.\\ \hline
3     & Verwaltungsrecht & Der Benutzer hat die Erlaubnis, das Objekt (Modul, Gruppe oder Datei) zu verändern oder zu löschen.\\ \hline
\end{tabularx}
\caption{Allgemeingültige Berechtigungsstufen.}
\label{tab:rights}
\end{table}

Jede Berechtigungsstufe beinhaltet alle tieferen Berechtigungen. Bei jedem Aufruf einer Seite, welche einen Benutzer, eine Datei, eine Gruppe oder ein Modul darstellt, wird überprüft, ob der angemeldete Benutzer die Leserechte besitzt. Bei unerlaubtem Zugriff wird eine entsprechende Fehlermeldung ausgegeben und der Zugriff bzw. das Anzeigen von Details verweigert. Bei jeder Anfrage, die Daten einer Datei, eines Benutzers, einer Gruppe oder eines Moduls zu verändern, wird überprüft, ob der angemeldete Benutzer die Rechte besitzt, um diese Änderungen vorzunehmen.\\

Ein Benutzer gilt als Mitglied einer Gruppe, wenn er mindestens Leserecht an einer Gruppe besitzt.
Ist der Benutzer in einer Gruppe, welche eine höhere Berechtigungsstufe hat, so übernimmt der Benutzer automatisch diese Berechtigung. Der Benutzer verliert diese Berechtigung, sobald er nicht mehr in dieser Gruppe ist (vergleiche \refreq{Gruppenrechte bearbeiten}).\\

Benutzer mit Verwaltungsrecht auf einem Objekt können anderen Benutzern und Gruppen auch Verwaltungsrecht auf dieses Objekt erteilen, aber ihnen dieses Recht nicht wieder entziehen. Diese Verwaltungsrechte können nur vom Ersteller des Objekts wieder entzogen werden.


\requirement{F0110}{An- und Abmelden}
Die Applikation bietet ein Formular, über welches sich ein Benutzer mit seiner E-Mail-Adresse und dem Passwort anmelden kann. Sind seine Eingaben korrekt und im System erfasst, wird er auf die Hauptseite umgeleitet. Ansonsten  wird dem Benutzer eine Fehlermeldung angezeigt und er kann es erneut versuchen. Zudem wird nun ein Link angezeigt, der es einem ermöglicht, sein Passwort zurückzusetzen (\refreq{Passwort zurücksetzen}).\\

Auf jeder Seite wird, sofern der Benutzer angemeldet ist, eine Schaltfläche angezeigt, über welche sich der Benutzer abmelden kann. Meldet sich ein Benutzer ab, wird er auf die Startseite umgeleitet. Nach einer 12-stündigen Inaktivität wird der Benutzer automatisch ausgeloggt.

\requirement{F0120}{Passwort zurücksetzen}
Nach Eingabe seiner E-Mail-Adresse und einem \gls{Captcha} kann der Benutzer sich ein neues Passwort generieren lassen. Dieses wird ihm dann per E-Mail zugestellt, ist ab sofort gültig und ersetzt das alte Passwort. Das generierte Passwort wird eine länge von 12 Zeichen haben und aus Gross-, Kleinbuchstaben, Zahlen und Sonderzeichen bestehen. 

\requirementsubsection{F02}{Benutzerverwaltung}
Benutzer stellen Personen dar. Jeder Benutzer hat individuelle Berechtigungen an Modulen, Gruppen und Dateien. Ein Benutzer kann auch selber neue Gruppen, Module und Dateien erstellen und Berechtigungen an diesen für andere Benutzer und Gruppen vergeben.

\requirement{F0200}{Registrierung}
Die Registrierung eines Benutzers erfolgt über ein Formular, welches folgende Angaben zwingend benötigt:
\begin{itemize}
\item Vorname
\item Nachname
\item E-Mail-Adresse
\item Passwort
\end{itemize}
Wenn alle Eingaben gemacht worden sind und die E-Mail-Adresse ein gültiges Format hat, wird dem Benutzer eine E-Mail zur Bestätigung gesendet, mit welcher er sein Konto aktivieren kann. Erst nach der Aktivierung des Kontos kann man sich auch anmelden. Sollte eine Eingabe nicht korrekt sein, wird das Formular erneut mit entsprechenden Fehlermeldungen angezeigt. In der E-Mail ist ein Link enthalten und eine kurze Beschreibung über den Nutzen dieses Links. Beim Öffnen der Seite, auf welche der Link verweist, wird der entsprechende Benutzer freigeschaltet. Dem Benutzer wird nun ein Loginformular und eine Bestätigung der Aktivierung angezeigt.
Um eine Erweiterung auf verschiedene Fachhochschulen und Universitäten zu erlauben, sind nur E-Mail-Adressen der ZHAW, also "`\texttt{@students.zhaw.ch}"' und "`\texttt{@zhaw.ch}"', zulässig.

\requirement{F0210}{Benutzerdaten ändern}
Jeder angemeldete Benutzer kann seine Benutzerdaten einsehen und über ein Formular ändern. Die editierbaren Benutzerdaten sind in der nachfolgenden Tabelle~\ref{tab:benutzer_eigenschaften} ersichtlich. Wird die E-Mail-Adresse geändert, wird wieder ein E-Mail mit einem Bestätigungs-Link versandt, mit welchem die neue E-Mail-Adresse validiert wird. Der Prozess der Validierung ist derselbe wie bei der Registrierung (\refreq{Registrierung}). Die E-Mail-Adresse wird erst geändert, wenn sie validiert wurde. 
Dem Benutzer ist es über das oben erwähnte Formular auch möglich, sein Benutzerkonto zu löschen. Löscht er sein Konto, so wird er automatisch abgemeldet (\refreq{An- und Abmelden}).

\begin{table}[H]
\begin{tabularx}{\textwidth}{|l|l|X|} \hline
\textbf{Eigenschaft} &\textbf{Editierbar} & \textbf{Beschreibung} \\ \hline
E-Mail-Addresse		& Ja 	& Die E-Mail-Adresse wird benötigt, um den Benutzer eindeutig zu identifizieren und um ihm allfällige Benachrichtigungen zukommen zu lassen.\\ \hline
Profilbild			& Ja 	& Ein optionales Profilbild.\\ \hline
Berechtigung 		& Ja 	& Alle Berechtigungen, die der Benutzer in Modulen, Gruppen und Dateien hat. Diese werden ihm gegeben und er kann sie nicht selber ändern.\\ \hline
Vorname 			& Ja 	& Der Vorname des Benutzers\\ \hline
Nachname			& Ja	& Der Nachname des Benutzers\\ \hline
Passwort			& Ja	& Das Passwort des Benutzers. Es darf aus allen Unicode-Zeichen, die korrekt als UTF-8 kodiert, sind bestehen. \\ \hline
\end{tabularx}
\caption{Eigenschaften eines Benutzers}
\label{tab:benutzer_eigenschaften}
\end{table}

\requirementsubsection{F03}{Gruppenverwaltung}
Gruppen werden verwendet, um mehreren Benutzern gleichzeitig Berechtigungen auf Module und Dateien zu geben. Es ist auch möglich, dass eine Gruppe Mitglied in einer anderen Gruppe ist. Eine Gruppe kann beispielsweise eine Klasse mit allen Studenten dieser Klasse sein.

\requirement{F0300}{Gruppe erstellen}
Jeder angemeldete Benutzer kann eine Gruppe erstellen, wodurch er automatisch das Verwaltungsrecht an der Gruppe erhält. Eine Gruppe kann nicht erstellt werden, wenn bereits eine andere Gruppe mit demselben Namen existiert, was durch die Ausgabe einer Fehlermeldung angezeigt wird.
Beim Erstellen einer Gruppe kann diese sogleich konfiguriert werden (\refreq{Gruppe bearbeiten}).

\requirement{F0310}{Gruppe bearbeiten}
Jeder angemeldete Benutzer, der ein Verwaltungsrecht an einer Gruppe besitzt, kann diese bearbeiten. Die zu bearbeitenden Eigenschaften einer Gruppe sind in der nachfolgenden Tabelle~\ref{tab:gruppe_eigenschaften} ersichtlich. Zudem kann jeder Benutzer mit Verwaltungsrecht die Gruppe löschen. Wird eine Gruppe gelöscht, werden den Mitgliedern der Gruppe alle Berechtigungen, die sie durch die Mitgliedschaft in der Gruppe an Modulen, Gruppen und Dateien geerbt haben, wieder entzogen.

\begin{table}[H]
\begin{tabularx}{\textwidth}{|l|l|X|} \hline
\textbf{Eigenschaft} &\textbf{Editierbar} & \textbf{Beschreibung} \\ \hline
Name				& Ja 	& Der Name der Gruppe, wie z.B. “IT15b Winterthur”. Der Name muss eindeutig sein. Es darf also keine andere Gruppe mit dem gleichen Namen geben.\\ \hline
Beschreibung		& Ja 	& Eine kurze Beschreibung der Gruppe.\\ \hline
Mitglieder			& Ja 	& Alle in der Gruppe enthaltenen Benutzer und Gruppen, inklusive den Berechtigungen, die sie in der Gruppe besitzen.\\ \hline
Ersteller	 		& Nein 	& Der Benutzer, welcher die Gruppe erstellt hat.\\ \hline
Erstelldatum		& Nein 	& Das Datum, an dem die Gruppe erstellt wurde.\\ \hline
\end{tabularx}
\caption{Eigenschaften einer Gruppe}
\label{tab:gruppe_eigenschaften}
\end{table}

\requirement{F0320}{Gruppenrechte bearbeiten}
Jeder angemeldete Benutzer mit Verwaltungsrecht \user{1} an einer Gruppe kann für andere Benutzer oder Gruppen \user{2} Berechtigungen erteilen.
Die zu vergebenden Berechtigungen sind in der nachfolgenden Tabelle~\ref{tab:gruppe_rechte} ersichtlich.
Der hinzugefügte Benutzer oder die Gruppe \user{2} übernimmt dann automatisch die Berechtigungen dieser Gruppe, welche diese in anderen Gruppen, Modulen oder Dateien besitzt.
Der Benutzer \user{1} kann ebenfalls Berechtigungen von Benutzern und Gruppen in dieser Gruppe entziehen oder ändern.
Werden Benutzern oder Gruppen alle Rechte in der Gruppe entzogen, verlieren sie alle Berechtigungen an Dateien, Modulen und anderen Gruppen, welche mit der Mitgliedschaft in der Gruppe einhergehen.


\begin{table}[H]
\begin{tabularx}{\textwidth}{|l|l|X|} \hline
\textbf{Stufe} & \textbf{Berechtigung}     & \textbf{Beschreibung} \\ \hline
0     & Keine Rechte     & Der Benutzer hat keine Berechtigung in der Gruppe, respektive ist nicht Mitglied in der Gruppe.\\ \hline
1     & Leserecht        & Der Benutzer ist Mitglied in der Gruppe.\\ \hline
%2     & Schreibrecht     & \\ \hline
3     & Verwaltungsrecht & Der Benutzer hat die Erlaubnis die Gruppe zu verändern oder zu löschen. Mögliche Änderungen sind: Name der Gruppe ändern, Beschreibung ändern und Berechtigungen vergeben.\\ \hline
\end{tabularx}
\caption{Berechtigungen in Gruppen}
\label{tab:gruppe_rechte}
\end{table}

\requirementsubsection{F04}{Modulverwaltung}
Module werden verwendet, um Dateien zu gruppieren, und sollen Module oder Fächer abbilden. Berechtigungen, die in Modulen für Gruppen und Benutzer vergeben werden, gelten für alle darin enthaltenen Dokumente.

\requirement{F0400}{Modul erstellen}
Jeder angemeldete Benutzer kann ein Modul erstellen, wodurch er sogleich das Verwaltungsrecht am Modul erhält. Ein Modul kann nicht erstellt werden, wenn bereits ein anderes Modul mit demselben Namen existiert, was durch die Ausgabe einer Fehlermeldung angezeigt wird.
Beim Erstellen eines Moduls kann dieses sogleich konfiguriert werden (\refreq{Modul bearbeiten}).

\requirement{F0410}{Modul bearbeiten}
Jeder angemeldete Benutzer, der ein Verwaltungsrecht an einem Modul besitzt, kann dieses bearbeiten. Die zu bearbeitenden Eigenschaften eines Moduls sind in nachfolgender Tabelle~\ref{tab:modul_eigenschaften} ersichtlich. Zudem kann jeder Benutzer mit Verwaltungsrecht das Modul löschen, was bewirkt, dass alle in das Modul hochgeladenen Dateien unwiderruflich gelöscht werden.

\begin{table}[H]
\begin{tabularx}{\textwidth}{|l|l|X|} \hline
\textbf{Eigenschaft} &\textbf{Editierbar} & \textbf{Beschreibung} \\ \hline
Name				& Ja 	& Der Name des Moduls. Zum Beispiel “Mathematik 1”. Dieser Name muss eindeutig sein. Es darf also kein anderes Modul mit dem gleichen Namen geben.\\ \hline
Beschreibung		& Ja 	& Eine kurze Beschreibung des Moduls.\\ \hline
Mitglieder			& Ja 	& Alle in dem Modul enthaltenen Benutzer und Gruppen, inklusive den Berechtigungen, die sie im Modul besitzen.\\ \hline
Ersteller	 		& Nein 	& Der Benutzer, welcher das Modul erstellt hat.\\ \hline
Erstelldatum		& Nein 	& Das Datum, an dem das Modul erstellt wurde.\\ \hline
\end{tabularx}
\caption{Eigenschaften eines Moduls}
\label{tab:modul_eigenschaften}
\end{table}

\requirement{F0420}{Modulrechte bearbeiten}
Jeder angemeldete Benutzer mit Verwaltungsrecht an einem Modul kann für andere Benutzer oder Gruppen Berechtigungen erteilen. Die zu vergebenden Berechtigungen sind in der nachfolgenden Tabelle~\ref{tab:modul_rechte} ersichtlich.
Der Benutzer mit Verwaltungsrecht kann ebenfalls Benutzern oder Gruppen ihre Berechtigungen, welche sie im Modul besitzen, entziehen oder ändern. Werden Benutzern oder Gruppen alle Rechte entzogen, sind sie nicht mehr Mitglied im Modul und haben somit auch keinen Zugriff auf die darin enthaltenen Dateien mehr.


\begin{table}[H]
\begin{tabularx}{\textwidth}{|l|l|X|} \hline
\textbf{Stufe} & \textbf{Berechtigung}     & \textbf{Beschreibung} \\ \hline
0     & Keine Rechte     & Der Benutzer hat keine Berechtigung in dem Modul und ist somit kein Mitglied des Moduls.\\ \hline
1     & Leserecht        & Der Benutzer hat die Erlaubnis, alle Dateien des Moduls zu betrachten.\\ \hline
2     & Schreibrecht     & Der Benutzer hat die Erlaubnis, neue Dateien in das Modul hochzuladen und hat Schreibrecht allen Dateien des Moduls (siehe \refreq{Datei bearbeiten}).\\ \hline
3     & Verwaltungsrecht & Der Benutzer hat die Erlaubnis, das Modul zu verändern oder zu löschen (\refreq{Modul bearbeiten}). Ebenfalls besitzt er Verwaltungsrecht an allen Dateien. \\ \hline
\end{tabularx}
\caption{Berechtigungen in Modulen}
\label{tab:modul_rechte}
\end{table}
\requirementsubsection{F05}{Dateimanagement}
Dateien repräsentieren zum Beispiel Notizen oder Zusammenfassungen. Diese Dateien werden in Module hochgeladen und sind von diesen abhängig. Das heisst, die Dateien können nicht ohne ein Modul existieren. Die Berechtigungen für Dateien werden vom Modul übernommen, in welches die Datei hochgeladen wird, können aber auch einzeln vergeben werden. Den Benutzern ist es auch möglich, Dateien zu bewerten und zu kategorisieren.

\requirement{F0500}{Datei hochladen}
Jeder Benutzer mit Schreibrecht an einem Modul kann Dateien in dieses Modul hochladen. Alle Dateiformate können hochgeladen werden und die maximalgrösse einer Datei beträgt 30 GB. Ein Upload von mehreren Dateien auf ein mal ist nicht vorgesehen. Es können keine Dateien mit dem gleichen Namen im selben Modul erstellt werden. Der Benutzer hat somit automatisch Verwaltungsrecht an der Datei. Abgesehen vom Verwaltungsrecht werden alle Berechtigungen der Benutzer oder Gruppen im Modul, in welches die Datei hochgeladen wird, auf die Datei übertragen.

\requirement{F0510}{Datei bearbeiten}
Jeder angemeldete Benutzer mit Schreibrecht für eine Datei kann deren Eigenschaften, exklusive den Berechtigungen und des Titels, ändern. Die zu bearbeitenden Eigenschaften einer Datei sind in der nachfolgenden Tabelle~\ref{tab:datei_eigenschaften} ersichtlich. Zudem kann dieser Benutzer die Datei austauschen. Ein Benutzer mit Verwaltungsrecht kann alle Eigenschaften bearbeiten und die Datei löschen. Das Löschen hat zur Folge, dass die Datei unwiderruflich aus dem System gelöscht wird und nicht mehr wiederhergestellt werden kann.

\begin{table}[H]
\begin{tabularx}{\textwidth}{|l|l|X|} \hline
\textbf{Eigenschaft} &\textbf{Editierbar} & \textbf{Beschreibung} \\ \hline
Titel				& Ja 	& Der Titel ist der Name der Datei, unter dem sie für Benutzer ersichtlich ist, wie zum Beispiel “Mathematik Zusammenfassung”. Dieser Titel muss im Modul, in dem die Datei hochgeladen wurde, eindeutig sein. Es darf also keine andere Datei im selben Modul mit dem gleichen Namen geben.\\ \hline
Beschreibung		& Ja 	& Eine kurze Beschreibung der Datei.\\ \hline
Berechtigungen		& Ja 	& Alle Berechtigungen, die für Benutzer und Gruppen dieser Datei erteilt wurden.\\ \hline
Kategorien			& Ja 	& Alle Kategorien, die der Datei zugewiesen wurden.\\ \hline
Bewertung			& Ja	& Jeder Benutzer kann eine Bewertung für die Datei abgeben und diese Bewertung gegebenenfalls auch wieder verändern.\\ \hline
Datei 				& Ja	& Die Datei selbst, wie zum Beispiel ein PDF-Dokument.\\ \hline
Dateityp			& Nein	& Der Dateityp der Datei, der von der hochgeladenen Datei abhängig ist.\\ \hline
Ersteller	 		& Nein 	& Der Benutzer, welcher die Datei erstellt hat.\\ \hline
Erstelldatum		& Nein 	& Das Datum, an dem die Datei erstellt wurde.\\ \hline
\end{tabularx}
\caption{Eigenschaften einer Datei}
\label{tab:datei_eigenschaften}
\end{table}

\requirement{F0520}{Dateirechte bearbeiten}
Jeder Benutzer mit Verwaltungsrecht für eine Datei kann auch Benutzern und Gruppen, die nicht Mitglied im entsprechenden Modul sind oder durch das Modul weniger Rechte an der Datei haben, Berechtigungen für diese Datei erteilen. Die zu vergebenden Berechtigungen und die daraus resultierenden Möglichkeiten für einen Benutzer oder eine Gruppe sind in der nachfolgenden Tabelle~\ref{tab:datei_rechte} ersichtlich.

\begin{table}[H]
\begin{tabularx}{\textwidth}{|l|l|X|} \hline
\textbf{Stufe} & \textbf{Berechtigung}     & \textbf{Beschreibung} \\ \hline
0     & Keine Rechte     & Der Benutzer hat keine Berechtigung auf diese Datei, respektive kann diese Datei nicht ansehen.\\ \hline
1     & Leserecht        & Der Benutzer hat die Erlaubnis, die Dateien zu betrachten und zu bewerten (\refreq{Datei bewerten}).\\ \hline
2     & Schreibrecht     & Der Benutzer hat die Erlaubnis, diese Datei mit einer neuen Datei auszutauschen (\refreq{Datei hochladen}). Er kann auch neue Kategorien hinzufügen und entfernen (\refreq{Datei kategorisieren}) und die Beschreibung der Datei ändern (\refreq{Datei bearbeiten}).\\ \hline
3     & Verwaltungsrecht & Der Benutzer hat die Erlaubnis, die Datei zu verändern oder zu löschen. Mögliche Änderungen sind in der obigen Tabelle~\ref{tab:datei_eigenschaften} ersichtlich.\\ \hline
\end{tabularx}
\caption{Berechtigungen in Modulen}
\label{tab:datei_rechte}
\end{table}

\requirement{F0530}{Datei anzeigen}
Jeder angemeldete Benutzer mit Leserecht für eine Datei kann Details zu dieser Datei in der Webseite ansehen. Wenn es das Dateiformat und die Grösse des Bildschirmes zulässt, wird die Datei im Browser angezeigt. Der Benutzer kann die Datei von der Webseite herunterladen und lokal abspeichern.

\requirement{F0540}{Datei bewerten}
Jeder angemeldete Benutzer mit Leserecht für eine Datei hat die Möglichkeit, diese Datei mit einem bis vier Sterne zu bewerten. Der Benutzer kann seine Bewertung nachträglich anpassen oder entfernen. \\
Die Wertung eines Dokumentes ist das Mittel aller Bewertungen und wird auf eine Kommastelle gerundet. Ausserdem wird immer angezeigt wie viele Bewertungen für ein Dokument schon abgegeben wurden.

\requirement{F0550}{Datei kategorisieren}
Jeder angemeldete Benutzer mit Schreibrecht für eine Datei kann zu dieser Datei Kategorien hinzufügen und entfernen. Um einer Unordnung betreffend der Kategorien vorzubeugen wird man bei der Kategorisierung Vorschläge von schon benutzten Begriffen angezeigt bekommen, welche man dann auswählen kann. Die Vorschläge werden aufgrund des Dokumententitels gesucht. Kategorien erleichtern die Suche nach Dateien. 

\requirementsubsection{F06}{Suche}
\requirement{F0600}{Suche}
Über ein Suchfeld kann jeder angemeldete Benutzer nach Objekten, also Dateien, Gruppen, Modulen und Benutzern suchen.\\
Gesucht wird in dieser Version der Applikation nur über Titel und Kategorien einer Datei. Es wird jedoch so entwickelt, dass es möglich wäre eine Suche über den Inhalt von Dokumenten zu implementieren.

Die Suchresultate können mit Hilfe von weiteren Such- und Filterkriterien sowohl vor der Suche als auch auf der Webseite, welche die Suchresultate darstellt, angepasst werden. Der Benutzer hat folgende Auswahlmöglichkeiten:
\begin{itemize}
\item Ein- bzw. Ausblenden von bestimmten Objekttypen, wie Dateien, Gruppen, Modulen oder Benutzern.
\item Ein- bzw. Ausblenden von Objekten, auf die der Benutzer keinen Lesezugriff hat. Auf diese Objekte kann der Benutzer Zugriff beantragen (\refreq{Zugriff beantragen}). 
\item Einschränkungen anhand der Eigenschaften von Objekten. Es können nur gemeinsame Eigenschaften von allen eingeblendeten Objekttypen eingeschränkt werden.
\end{itemize}
Dem Benutzer ist es auch möglich, die Suchresultate nach gemeinsamen Eigenschaften der Resultate zu sortieren. Ohne benutzerdefinierte Sortierung werden Dateien mit einer guten und aktuellen Bewertung für den Benutzer weiter oben in der Liste und somit besser sichbar dargestellt.\\
Welcher Suchalgorithmus verwendet wird, ist noch nicht bestimmt. Es wird jedoch ein bewährter Algorithmus eingesetzt, um Stabilität, Effizienz und Verlässlichkeit zu garantieren.

\requirement{F0610}{Zugriff beantragen}
Jeder angemeldete Benutzer kann für Module, Gruppen und Dateien, die er über die Suche findet, auf welche er aber keinen Zugriff hat, Berechtigungen beantragen. Dies geschieht über ein Fenster in dem er die gewünschte Berechtigung und den Grund, wieso er diese Berechtigung benötigt, angibt. Danach wird allen Verwaltern des Objekts ein E-Mail mit den oben erwähnten Angaben gesendet.

\requirement{F0620}{Zugriff freigeben}
Der Ersteller eines Objektes erhält im Falle einer Zugriffsanfrage (\refreq{Zugriff beantragen}) eine E-Mail. Darin enthalten ist der Name des Objektes, auf das der Zugriff beantragt wird, der Antragsteller, die benötigte Berechtigung und der Grund, der zum Wunsch nach Zugriff geführt hat. Über einen Link im E-Mail kann der Ersteller dann den Zugriff freigeben.
