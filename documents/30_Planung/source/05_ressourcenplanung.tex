\section{Ressourcenplanung}
\subsection{Meilensteine}
Das Team hat beschlossen, für das gesamte Projekt einen Projektleiter zu bestimmen. Dies wurde aus meheren Gründen so entschieden.
Einerseits ist so immer klar wer den Überblick über den Projektstand haben muss. Andererseits eliminieren man so das Risiko von Wissens-Verluste bei der Übergabe des Projektstandes zwischen Projektleitern.\\

\begin{table}[H]
\centering
\begin{tabularx}{\textwidth-1cm}{|l|l|X|} \hline
\textbf{Meilenstein}	& \textbf{Verantwortlichkeit} &	\textbf{Erwartung} \\ \hline
\textbf{M1}	&Sascha Bergmann	&Vorschau GUI, Datenbank (ER-Schema) \\  \hline
\textbf{M2}	&Sascha Bergmann	&Design und DB umgesetzt; mind. Ein Hauptprozess vollständig umgesetzt \\ \hline
\textbf{M3}	&Sascha Bergmann	&Hauptprozesse umgesetzt; Codierungsstil/Modularisierung \\ \hline
\textbf{M4}	&Sascha Bergmann	&Abnahmetests; Gruppenspezifischer Schwerpunkt \\ \hline
\textbf{M5}	&Sascha Bergmann	&Präsentation der Arbeit \\ \hline
\end{tabularx}
\caption{Definition der Meilensteine.}
\end{table}

Eine komplette Auflistung der Arbeitspackete und mit den zugewiesenen Meilensteine ist im Anhang~\ref{sub:arbeitspakete_und_aufwandschaetzung} zu finden.  

\iffalse
  \begin{ganttchart}{1}{12}
    \gantttitle{Projektplan}{12} \\
    \gantttitlelist{1,...,12}{1} \\
    \ganttmilestone{Meilenstein M1}{4}  \ganttnewline
    \ganttmilestone{Meilenstein M2}{5}  \ganttnewline
    \ganttmilestone{Meilenstein M3}{6}  \ganttnewline
    \ganttmilestone{Meilenstein M4}{7} 
  \end{ganttchart}
\fi

\subsection{Auslastung}
Die Auslastung ist bei allen Personen sehr ähnlich. Für den Projektleiter Sascha Bergmann sind im Durchschnitt weniger Arbeitsstunden eingeplant. Dies ist absichtlich so gelöst, da der Projektleiter noch Zeit benötigt um administrative Arbeiten auszuführen, wie z.B. zur Vorbereitung von Meilenstein-Sitzungen, um Arbeitsstände zu überprüfen und um eventuelle Planungsänderungen vorzunehmen.

\begin{figure}[H]
\centering
\pgfplotstableread{
MS    bergmansas     kunzlio    muellcy1    nguyeda
1     13.75          13.75      18.75       13.75
2     10             12         11          20
3     12             17         16          10
4     12.5           12.5       12.5        12.5
}\datatable

\begin{tikzpicture}
\begin{axis}[
	width=\linewidth-2cm,
	xbar stacked,
	area legend,
	ytick=data,
	yticklabels={Aufwand M1,
				Aufwand M2,
				Aufwand M3,
				Aufwand M4,
				%\emph{Total}
				},
	legend style={
		legend columns=2,
		at={(xticklabel cs:0.5, 20)},
		anchor=north,
		draw=none
	},
	xlabel={Aufwand in Stunden},
	xticklabel pos=lower,
	minor x tick num=1,
	xmin=0,
	xmax=65,
	bar width=4mm,
	y=7.5mm,
	enlarge x limits={abs=0},
	enlarge y limits={abs=0.7},
	grid=major,
	y dir=reverse,
    point meta=explicit,
    %calculate offset/.code={
    %    \pgfkeys{/pgf/fpu=true,/pgf/fpu/output format=fixed}
    %    \pgfmathsetmacro\testmacro{(\pgfplotspointmeta*10^\pgfplots@data@scale@trafo@EXPONENT@x)/2*\pgfplots@x@veclength)}
    %    \pgfkeys{/pgf/fpu=false}
    %},
    %every node near coord/.style={
    %    /pgfplots/calculate offset,
    %    yshift=-\testmacro
    %},
	%nodes near coords,
	%nodes near coords align=center
]
\addplot[bergmansas,fill=bergmansas] table [y={MS}, x=bergmansas] \datatable;
\addplot[kunzlio,fill=kunzlio] table [y={MS}, x=kunzlio] \datatable;
\addplot[muellcy1,fill=muellcy1] table [y={MS}, x=muellcy1] \datatable;
\addplot[nguyeda,fill=nguyeda] table [y={MS}, x=nguyeda] \datatable;
\legend{S. Bergmann, L. Kunz, C. Müller, D. Nguyen}
\end{axis}
\end{tikzpicture}
\caption{Auslastung pro Meilenstein.}
\end{figure}



\begin{figure}[H]
\centering
\begin{tikzpicture}
\centering
\begin{axis}[
	width=\linewidth-2cm,
	xbar stacked,
	area legend,
	ytick=data,
	yticklabels={S. Bergmann,
				L. Kunz,
				C. Müller,
				D. Nguyen,
				\emph{Total}
				},
	xlabel={Aufwand in Stunden},
	xticklabel pos=lower,
	minor x tick num=1,
	xmin=0,
	xmax=65,
	bar width=4mm,
	y=7.5mm,
	enlarge x limits={abs=0},
	enlarge y limits={abs=0.7},
	grid=major,
	y dir=reverse
]
\addplot[bergmansas,fill=bergmansas, x={label}, y={bergmansas}] coordinates
% Transfer
{%(13,0)
(48.25, 0)
(0,1)
(0,2)
(0,3)
};

\addplot[kunzlio,fill=kunzlio] coordinates
{
(0,0)
(55.25,1)
(0,2)
(0,3)
};

\addplot[muellcy1,fill=muellcy1] coordinates
{
(0,0)
(0,1)
(58.25,2)
(0,3)
%(18.75,0)
%(11,1)
%(16,2)
%(12.5,3)
%(,4)
};

\addplot[nguyeda, fill=nguyeda] coordinates
{
(0,0)
(0,1)
(0,2)
(56.25,3)
%(,4)
};

%\legend{Sascha Bergmann, Lion Kunz, Cyril Müller, Dang Thien Nguyen}

\end{axis}
\end{tikzpicture}
\caption{Gesamte Auslastung pro Person.}
\end{figure}

\iffalse % auskommentieren der jpg diagramme
\begin{figure}[H]
\centering
\includegraphics[width=0.7\textwidth]{graphics/auslastung_meilensteine.png}
\caption{Aufwand pro Meilenstein.}
\end{figure}


\begin{figure}[H]
\centering
\includegraphics[width=0.7\textwidth]{graphics/auslastung_total.png}
\caption{Aufwand pro Person.}
\end{figure}
\fi