
\documentclass{avocado}


\usepackage[ngerman]{babel}
\usepackage[german=quotes]{csquotes}
\usepackage{float}
%\usepackage{bibgerm}
\usepackage{amsmath}
\usepackage{tabularx}
\usepackage{graphicx}
\usepackage{pdflscape}
\usepackage{xstring}
\usepackage{caption}
\usepackage[backend=biber, style=ieee]{biblatex}
\addbibresource{sources.bib}
%\usepackage[fixlanguage]{babelbib}
%\selectbiblanguage{german}
%\bibliographystyle{ieeetr}
%\addbibresource{sources}
\newcommand{\mail}[1]{\href{mailto:#1}{#1}}
\setlength{\parindent}{0cm}

\newcommand{\titel}{Avocado Share}
\newcommand{\shorttitel}{}
\newcommand{\doctype}{Pflichtenheft}
\newcommand{\untertitel}{Studentenplattform zum Know-How-Transfer}
\newcommand{\datum}{\today}
\newcommand{\team}{Gruppe 13}
\newcommand{\autorA}{Bergmann Sascha}
\newcommand{\autorB}{Kunz Lion}
\newcommand{\autorC}{Ngueyen Dang Thien}
\newcommand{\autorD}{Müller Cyril}
\newcommand{\autorE}{}
\newcommand{\ort}{Winterthur}
\newcommand{\dozent}{}
\newcommand{\betreuer}{}
\newcommand{\version}{1.6}

\hypersetup{
    bookmarks=false,    % show bookmarks bar?
    pdftitle={\titel - \doctype},    % title
    pdfauthor={\autorA, \autorB, \autorC, \autorD},  % author
    pdfsubject={Funktionale und nicht-funktionale Anforderungen}, % subject of the document
    pdfkeywords={\titel, \doctype, \team, Version \version, PSIT, IT15b}, % list of keywords
    colorlinks=true,        % false: boxed links; true: colored links
    linkcolor=blue!30!black,% color of internal links
    citecolor=black,        % color of links to bibliography
    filecolor=black,        % color of file links
    urlcolor=black,         % color of external links
    linktoc=page            % only page is linked
}%

%\project{Avocado Share}
\title{\title}
\author{\autorA \and \autorB \and \autorC \and \autorD \and \autorD}

\newcommand{\printrequirement}[2]{%
    \texttt{/#1/} #2% 
}

\newcommand{\requirementsection}[2]{%
    \section{\printrequirement{#1xxxx}{#2}}%
}


\newcommand{\requirementsubsection}[2]{%
    \subsection{\printrequirement{#1xx}{#2}}% 
}


\newcommand{\requirement}[2]{%
% #1: The requirement number e.g. "T0100"
% #2: The requirement title
    \subsubsection*{\printrequirement{#1}{#2}}%
    \expandafter\def\csname RequirementName#1\endcsname{#2}
    \expandafter\def\csname RequirementNumber#2\endcsname{#1}
    \label{subsub:requirement_#2}
}

\newcommand{\user}[1]{$U_#1$}
\newcommand{\group}[1]{$G_#1$}

\newenvironment{abnahmefall}[1][]{%
%begin{abnahmefall}
\tabularx{\textwidth}{|lX|} \hline
\ifthenelse{\equal{#1}{}}
{}{%
& \textbf{#1} \\ \hline
}%
\ignorespaces}{
%end{abnahmefall}
\endtabularx\\[1em]
}

\newcommand{\ausgangssituation}[1]{\textbf{Ausgangssituation} &\ignorespaces #1 \\ \hline}
\newcommand{\ereignis}[1]{\textbf{Ereignis} &\ignorespaces #1 \\ \hline }
\newcommand{\ergebnis}[1]{\textbf{Erwartetes Ergebnis} &\ignorespaces #1 \\ \hline}

% Refer to a requirement by it's title
\newcommand{\refreq}[1]{\nameref{subsub:requirement_#1}}
\newcommand{\tbf}{\emph{Noch zu definieren.}}

\newcommand{\abnahmekriterium}[1]{%
\StrLeft{\csname RequirementNumber#1\endcsname}{1}[\requirementsectioncharacter]%
\StrBehind{\csname RequirementNumber#1\endcsname}{\requirementsectioncharacter}[\plainrequirementnumber]%
\subsubsection*{\printrequirement{T\plainrequirementnumber}{#1}}%
}


\usepackage[toc, xindy]{glossaries}
%\newglossary[glignoredl]{ignored}{glignored}{glignoredin}{Ignored Glossary}


\makeglossaries

\begin{document}
% title page -------------------------------------------------------------------
\thispagestyle{plain}

\begin{titlepage}
    \begin{flushleft}
        \vspace*{1cm}
        \textbf{
            \hspace{-0.12cm}\LARGE{\doctype}\\
            \Huge{\titel}\\
            \vspace{0.5cm}
            \large{\untertitel}\\
            \vspace{1.5cm}
            \large{\team\\}
        }
            \large{\autorA, \autorB,\\\autorC, \autorD}\\
        \vspace{1cm}
        \vfill
        \large{
            \iffalse % We don't need a dozent or a betreuer
                \hspace{-0.83cm} \includegraphics{images/zhaw_logo_full}\\
                \line(1,0){165} 

                \vspace{0.5cm}
                Auftraggeber:\\ \dozent \\
                Betreuer:\\ \betreuer \\
                \vspace{0.5cm}
            \fi
            \ort, \datum \hfill Version: \version
        }
    \end{flushleft}
\end{titlepage}
% end of titlepage -------------------------------------------------------------
\section*{Dokumentenhistorie}
\begin{tabularx}{\linewidth}{|l|r|X|} \hline
Version & \multicolumn{1}{l|}{Datum} 			& Anpassungen \\ \hline
0.1 & 05.12.2015 & Dokumentstruktur und erster Entwurf \\ \hline
0.2 & 06.12.2015 & Überarbeitung des Inhaltes und der Diagramme \\ \hline
\end{tabularx}

\clearpage

\vfill
\section*{Unterschriften}
\begin{minipage}[t][5cm][t]{0.45\linewidth}
\subsection*{Auftraggeber}
Prof. Dr. Max Lemmenmeier \\
ZHAW, Dept. Linguistik \\
%LCC Language Competence Centre \\
%Büro SF 03.09 \\
Theaterstr. 17 \\
%Postfach \\
8400 Winterthur \\
%Tel. 0041 58 934 60 73 \\
\mail{max.lemmenmeier@zhaw.ch} \\
\vfill \hrule
\end{minipage} \hfill
\begin{minipage}[t][5cm][t]{0.45\linewidth}
\subsection*{Auftragnehmer}
Leiter Projektgruppe \\
Sascha Bergmann \\
ZHAW, Bachelorstudent ICT \\
\mail{bergmsas@students.zhaw.ch} \\
\vfill \hrule
\end{minipage} \\
%\begin{minipage[t]{0.45\linewidth}
%\subsection*{Auftraggeber}
%\end{minipage}
\clearpage

\tableofcontents
\clearpage
\section{Einleitung}
Die Projektplanung ist ein wichtiger Bestandteil des Projektmanagements. 
In diesem Dokument wird ein Grundstein gelegt, um eine genaue Termin- und Aufwandplanung des Projektes
zu ermöglichen. 
Die Anforderungen, wie sie im Pflichtenheft beschrieben sind, wurden in Arbeitspaketen gegliedert, welche
im Laufe der nächsten Projektphasen ausgeführt werden. Im Netzplan werden diese Pakete graphisch dargelegt.
Zusätzlich zeigt dieses Dokument sowohl den gesamten Aufwand, als auch die Auslastung der einzelnen Teammitglieder auf.

\section{Netzplan}
Im folgenden Netzplan sind alle Arbeitspakete der Entwicklung des Avocado-Share dargestellt. 
Um sicherzustellen, dass jedes Arbeitspaket termingerecht und vollständig ausgeführt wird, haben wir beschlossen, neben dem Verantwortlichen auch eine Person zu bestimmen, welche eine Aufgabe ausführt.\\

Durch eine Planungsphase zu Beginn des Projektes, ist es uns möglich eine klare Schichtentrennung zu machen.
Da wir in der Planung die Schnittstellen zwischen den Schichten ist es uns möglich die Entwicklungsschritte danach
alle parallel auszuführen. Dadurch erhalten wir zwar einen kleinen „bottle-neck“ beim Erstellen der Grobstruktur und des
Grundgerüstes, doch es bringt uns viel Freiheit in den folgenden Paketen. So können wir besser und einfacher allfällige
Verzögerungen und Ausfälle reagieren.
Da wir eine technisch saubere Lösung haben wollen, ist es uns wichtig, dass die Spezialisten eines Bereiches in unserem Team auch entweder Verantwortlicher oder Ausführender eines Arbeitspaketes sind. \\

In jedem Arbeitspaket inbegriffen ist, wo möglich, auch das Testing.
Bei Arbeitspaketen in welchen Java-Code geschrieben wurde, sollten Unit-Tests geschrieben werden.
Bei Web- und UI-Paketen sollte ein kurzes Test-Protokoll geschrieben werden, um sicherzustellen, dass alle Funktionen auch korrekt funktionieren.
So kann beim Arbeitspaket Testing schnell und einfach alles nochmals getestet werden.\\

Design- und Planungspakete haben als „Test“ ein Review mit dem gesamten Team.
\section{Ressourcenplanung}
\subsection{Meilensteine}
Unser Team hat sich darauf geeinigt, dass der Projektleiter über die gesamte Zeit bei derselben Person bleibt.
So ist immer klar wer den Überblick haben muss und es gibt keine Wissens-Verluste bei der Übergabe des Projektstandes zwischen Projektleitern. 

\begin{tabularx}{\textwidth-2cm}{|l|l|X|} \hline
\textbf{Meilenstein}	& \textbf{Verantwortlichkeit} &	\textbf{Erwartung} \\ \hline
\textbf{M1}	&Sascha Bergmann	&Vorschau GUI, Datenbank (ER-Schema) \\  \hline
\textbf{M2}	&Sascha Bergmann	&Design und DB umgesetzt; mind. Ein Hauptprozess vollständig umgesetzt \\ \hline
\textbf{M3}	&Sascha Bergmann	&Hauptprozesse umgesetzt; Codierungsstil/Modularisierung \\ \hline
\textbf{M4}	&Sascha Bergmann	&Abnahmetests; Gruppenspezifischer Schwerpunkt \\ \hline
\textbf{M5}	&Sascha Bergmann	&Präsentation der Arbeit \\ \hline
\end{tabularx}

\end{document}